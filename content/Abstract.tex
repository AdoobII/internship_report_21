\chapter{Abstract}\label{Abstract}

Current 3D laser line scanners have precision in the range of a micrometer. These scanners work on the principle of laser triangulation and use a camera chip in the receive path. The captured pixel data is then processed on an FPGA to generate 3D profile data. In order to do this, the laser line, as seen by the camera, must be extracted from the pixel data. For this project, several methods have been proposed. One of these methods employs an FIR filter to calculate the derivative of the incoming pixel stream orthogonally to the laser line direction. Afterwards, the zero crossing of this derivative is detected. The position of the zero crossing marks the position of the laser line in the camera image. From this position, the distance of the laser scanner to the scanned object can be derived.

This project was implemented using VHDL for the digital modules. In addition to that, Python tools were developed to assist in system modeling, simulation and test vector generation. The report highlights the proposed solution, an overview of the scripts and tools developed to assist in the system architecture design, a comprehensive look at each module, and finally a look at how the system can be improved.
